\documentclass{standalone}
\usepackage{color}
\usepackage{tikz}
\usepackage{pgfplots}
\usepackage{pgf-umlsd}
\usepackage{tikz}

\pgfplotsset{compat=1.16} 


%%%%%%%%%%%%%
%Figure adapted from: http://www.texample.net/tikz/examples/neural-network/
%Neuron style from: Yoon, Jaehong, Eunho Yang, Jeongtae Lee, and Sung Ju Hwang. "Lifelong learning with dynamically expandable networks." arXiv preprint arXiv:1708.01547 (2017).
%%%%%%%%%%%%%

\begin{document}
\pagestyle{empty}

\def\layersep{2.5cm}

\begin{tikzpicture}[-stealth, semithick]
% \tikzstyle{every pin edge}=[<-,shorten <=1pt]
\tikzstyle{annot} = [text width=6em, text centered]
    
\newcommand{\newneuron}[5]{
    \def\name{#1} % name of the neuron
    \def\x{#2} % x coordinate of the neuron
	\def\y{#3} % y coordinate of the neuron
	\def\color{#4} % color of the neuron
	\def\size{#5} % size of the neuron
	\node[draw=black, very thick, fill=\color!50, circle, minimum size=\size] (\name) at (\x, \y) {};
	\node[draw=black, semithick, circle, minimum size=0.75*\size] at (\x, \y) {};
}

% % Draw the input layer nodes
\foreach \na / \z in {2,...,4}
\newneuron{I-\na}{0}{-\z}{green}{15pt};

% % Draw the hidden layer nodes
\foreach \na / \z in {1,...,5}
\newneuron{H-\na}{\layersep}{-\z}{blue}{15pt};

% % Draw the output layer node
% \node[output neuron,pin={[pin edge={->}]right:Output}, right of=H-3] (O) {};
\newneuron{O}{2*\layersep}{-3}{red}{15pt};

% Connect every node in the input layer with every node in the
% hidden layer.
\foreach \source in {2,...,4}
\foreach \dest in {1,...,5}
    \draw (I-\source) -- (H-\dest);

% Connect every node in the hidden layer with the output layer
\foreach \source in {1,...,5}
\draw(H-\source) -- (O);

% Annotate the layers
\node[annot,above of=H-1, node distance=1cm] (hl) {Hidden layer};
\node[annot,left of=hl, node distance=\layersep] {Input layer};
\node[annot,right of=hl, node distance=\layersep] {Output layer};
\end{tikzpicture}
% End of code
\end{document}